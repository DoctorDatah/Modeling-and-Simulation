% Options for packages loaded elsewhere
\PassOptionsToPackage{unicode}{hyperref}
\PassOptionsToPackage{hyphens}{url}
%
\documentclass[
]{article}
\usepackage{lmodern}
\usepackage{amssymb,amsmath}
\usepackage{ifxetex,ifluatex}
\ifnum 0\ifxetex 1\fi\ifluatex 1\fi=0 % if pdftex
  \usepackage[T1]{fontenc}
  \usepackage[utf8]{inputenc}
  \usepackage{textcomp} % provide euro and other symbols
\else % if luatex or xetex
  \usepackage{unicode-math}
  \defaultfontfeatures{Scale=MatchLowercase}
  \defaultfontfeatures[\rmfamily]{Ligatures=TeX,Scale=1}
\fi
% Use upquote if available, for straight quotes in verbatim environments
\IfFileExists{upquote.sty}{\usepackage{upquote}}{}
\IfFileExists{microtype.sty}{% use microtype if available
  \usepackage[]{microtype}
  \UseMicrotypeSet[protrusion]{basicmath} % disable protrusion for tt fonts
}{}
\makeatletter
\@ifundefined{KOMAClassName}{% if non-KOMA class
  \IfFileExists{parskip.sty}{%
    \usepackage{parskip}
  }{% else
    \setlength{\parindent}{0pt}
    \setlength{\parskip}{6pt plus 2pt minus 1pt}}
}{% if KOMA class
  \KOMAoptions{parskip=half}}
\makeatother
\usepackage{xcolor}
\IfFileExists{xurl.sty}{\usepackage{xurl}}{} % add URL line breaks if available
\IfFileExists{bookmark.sty}{\usepackage{bookmark}}{\usepackage{hyperref}}
\hypersetup{
  pdftitle={R Notebook},
  hidelinks,
  pdfcreator={LaTeX via pandoc}}
\urlstyle{same} % disable monospaced font for URLs
\usepackage[margin=1in]{geometry}
\usepackage{color}
\usepackage{fancyvrb}
\newcommand{\VerbBar}{|}
\newcommand{\VERB}{\Verb[commandchars=\\\{\}]}
\DefineVerbatimEnvironment{Highlighting}{Verbatim}{commandchars=\\\{\}}
% Add ',fontsize=\small' for more characters per line
\usepackage{framed}
\definecolor{shadecolor}{RGB}{248,248,248}
\newenvironment{Shaded}{\begin{snugshade}}{\end{snugshade}}
\newcommand{\AlertTok}[1]{\textcolor[rgb]{0.94,0.16,0.16}{#1}}
\newcommand{\AnnotationTok}[1]{\textcolor[rgb]{0.56,0.35,0.01}{\textbf{\textit{#1}}}}
\newcommand{\AttributeTok}[1]{\textcolor[rgb]{0.77,0.63,0.00}{#1}}
\newcommand{\BaseNTok}[1]{\textcolor[rgb]{0.00,0.00,0.81}{#1}}
\newcommand{\BuiltInTok}[1]{#1}
\newcommand{\CharTok}[1]{\textcolor[rgb]{0.31,0.60,0.02}{#1}}
\newcommand{\CommentTok}[1]{\textcolor[rgb]{0.56,0.35,0.01}{\textit{#1}}}
\newcommand{\CommentVarTok}[1]{\textcolor[rgb]{0.56,0.35,0.01}{\textbf{\textit{#1}}}}
\newcommand{\ConstantTok}[1]{\textcolor[rgb]{0.00,0.00,0.00}{#1}}
\newcommand{\ControlFlowTok}[1]{\textcolor[rgb]{0.13,0.29,0.53}{\textbf{#1}}}
\newcommand{\DataTypeTok}[1]{\textcolor[rgb]{0.13,0.29,0.53}{#1}}
\newcommand{\DecValTok}[1]{\textcolor[rgb]{0.00,0.00,0.81}{#1}}
\newcommand{\DocumentationTok}[1]{\textcolor[rgb]{0.56,0.35,0.01}{\textbf{\textit{#1}}}}
\newcommand{\ErrorTok}[1]{\textcolor[rgb]{0.64,0.00,0.00}{\textbf{#1}}}
\newcommand{\ExtensionTok}[1]{#1}
\newcommand{\FloatTok}[1]{\textcolor[rgb]{0.00,0.00,0.81}{#1}}
\newcommand{\FunctionTok}[1]{\textcolor[rgb]{0.00,0.00,0.00}{#1}}
\newcommand{\ImportTok}[1]{#1}
\newcommand{\InformationTok}[1]{\textcolor[rgb]{0.56,0.35,0.01}{\textbf{\textit{#1}}}}
\newcommand{\KeywordTok}[1]{\textcolor[rgb]{0.13,0.29,0.53}{\textbf{#1}}}
\newcommand{\NormalTok}[1]{#1}
\newcommand{\OperatorTok}[1]{\textcolor[rgb]{0.81,0.36,0.00}{\textbf{#1}}}
\newcommand{\OtherTok}[1]{\textcolor[rgb]{0.56,0.35,0.01}{#1}}
\newcommand{\PreprocessorTok}[1]{\textcolor[rgb]{0.56,0.35,0.01}{\textit{#1}}}
\newcommand{\RegionMarkerTok}[1]{#1}
\newcommand{\SpecialCharTok}[1]{\textcolor[rgb]{0.00,0.00,0.00}{#1}}
\newcommand{\SpecialStringTok}[1]{\textcolor[rgb]{0.31,0.60,0.02}{#1}}
\newcommand{\StringTok}[1]{\textcolor[rgb]{0.31,0.60,0.02}{#1}}
\newcommand{\VariableTok}[1]{\textcolor[rgb]{0.00,0.00,0.00}{#1}}
\newcommand{\VerbatimStringTok}[1]{\textcolor[rgb]{0.31,0.60,0.02}{#1}}
\newcommand{\WarningTok}[1]{\textcolor[rgb]{0.56,0.35,0.01}{\textbf{\textit{#1}}}}
\usepackage{graphicx,grffile}
\makeatletter
\def\maxwidth{\ifdim\Gin@nat@width>\linewidth\linewidth\else\Gin@nat@width\fi}
\def\maxheight{\ifdim\Gin@nat@height>\textheight\textheight\else\Gin@nat@height\fi}
\makeatother
% Scale images if necessary, so that they will not overflow the page
% margins by default, and it is still possible to overwrite the defaults
% using explicit options in \includegraphics[width, height, ...]{}
\setkeys{Gin}{width=\maxwidth,height=\maxheight,keepaspectratio}
% Set default figure placement to htbp
\makeatletter
\def\fps@figure{htbp}
\makeatother
\setlength{\emergencystretch}{3em} % prevent overfull lines
\providecommand{\tightlist}{%
  \setlength{\itemsep}{0pt}\setlength{\parskip}{0pt}}
\setcounter{secnumdepth}{-\maxdimen} % remove section numbering

\title{R Notebook}
\author{}
\date{\vspace{-2.5em}}

\begin{document}
\maketitle

\begin{Shaded}
\begin{Highlighting}[]
\KeywordTok{library}\NormalTok{(MASS)}

\CommentTok{# Loading Data}
\NormalTok{covid_data =}\StringTok{ }\KeywordTok{read.csv}\NormalTok{(}\StringTok{"./nCoV_simple.csv"}\NormalTok{)}\OperatorTok{$}\NormalTok{days}

\CommentTok{# Function to fit dist and return parameter values}
\NormalTok{fit_distribution <-}\StringTok{ }\ControlFlowTok{function}\NormalTok{(dist) \{}
  
\NormalTok{    parameters =}\StringTok{ }\KeywordTok{fitdistr}\NormalTok{(covid_data,dist)}
    \KeywordTok{return}\NormalTok{(parameters)}
\NormalTok{\}}

\CommentTok{# Function for log_liklehood}
\NormalTok{calculate_log_likelihood <-}\StringTok{ }\ControlFlowTok{function}\NormalTok{(parameters,dist)\{}
\NormalTok{    p1 =}\StringTok{ }\KeywordTok{as.numeric}\NormalTok{(parameters}\OperatorTok{$}\NormalTok{estimate[}\DecValTok{1}\NormalTok{])}
\NormalTok{    p2 =}\StringTok{ }\KeywordTok{as.numeric}\NormalTok{(parameters}\OperatorTok{$}\NormalTok{estimate[}\DecValTok{2}\NormalTok{])}
    \KeywordTok{print}\NormalTok{(dist)}
    \KeywordTok{log}\NormalTok{(}\KeywordTok{prod}\NormalTok{(}\KeywordTok{get}\NormalTok{(dist)(covid_data,p1, p2)))}
  
\NormalTok{\}}
\end{Highlighting}
\end{Shaded}

In class, we used the nCoV simple dataset and fit normal, lognormal and
gamma models. Fit a Weibull model to the same dataset and answer the
following questions.

\begin{enumerate}
\def\labelenumi{(\arabic{enumi})}
\tightlist
\item
  What are the parameter estimates of the Weibull model?
\end{enumerate}

\begin{Shaded}
\begin{Highlighting}[]
\NormalTok{parameters_ =}\StringTok{ }\KeywordTok{fit_distribution}\NormalTok{(}\StringTok{"Weibull"}\NormalTok{)}
\NormalTok{parameters_}
\end{Highlighting}
\end{Shaded}

\begin{verbatim}
##      shape       scale  
##   1.8192496   4.6111525 
##  (0.1992127) (0.3768674)
\end{verbatim}

\begin{enumerate}
\def\labelenumi{(\arabic{enumi})}
\setcounter{enumi}{1}
\tightlist
\item
  What is the log likelihood of the Weibull model? Also, write down the
  log likelihood of the normal, lognormal, and gamma models. Based on
  the log likelihood, which model is the best fit?
\end{enumerate}

\begin{Shaded}
\begin{Highlighting}[]
\KeywordTok{calculate_log_likelihood}\NormalTok{(}\DataTypeTok{parameters =}\NormalTok{ parameters_, }\StringTok{"dweibull"}\NormalTok{)}
\end{Highlighting}
\end{Shaded}

\begin{verbatim}
## [1] "dweibull"
\end{verbatim}

\begin{verbatim}
## [1] -109.5789
\end{verbatim}

\begin{Shaded}
\begin{Highlighting}[]
\CommentTok{# Normal }
\NormalTok{parameters_n =}\StringTok{ }\KeywordTok{fit_distribution}\NormalTok{(}\StringTok{"normal"}\NormalTok{)}
\KeywordTok{calculate_log_likelihood}\NormalTok{(}\DataTypeTok{parameters =}\NormalTok{ parameters_n, }\StringTok{"dnorm"}\NormalTok{)}
\end{Highlighting}
\end{Shaded}

\begin{verbatim}
## [1] "dnorm"
\end{verbatim}

\begin{verbatim}
## [1] -112.9528
\end{verbatim}

\begin{Shaded}
\begin{Highlighting}[]
\CommentTok{# log Normal }
\NormalTok{parameters_ln =}\StringTok{ }\KeywordTok{fit_distribution}\NormalTok{(}\StringTok{"lognormal"}\NormalTok{)}
\KeywordTok{calculate_log_likelihood}\NormalTok{(}\DataTypeTok{parameters =}\NormalTok{ parameters_ln, }\StringTok{"dlnorm"}\NormalTok{)}
\end{Highlighting}
\end{Shaded}

\begin{verbatim}
## [1] "dlnorm"
\end{verbatim}

\begin{verbatim}
## [1] -115.7273
\end{verbatim}

\begin{Shaded}
\begin{Highlighting}[]
\CommentTok{# gamma}
\NormalTok{parameters_g =}\StringTok{ }\KeywordTok{fit_distribution}\NormalTok{(}\StringTok{"gamma"}\NormalTok{)}
\end{Highlighting}
\end{Shaded}

\begin{verbatim}
## Warning in densfun(x, parm[1], parm[2], ...): NaNs produced
\end{verbatim}

\begin{Shaded}
\begin{Highlighting}[]
\KeywordTok{calculate_log_likelihood}\NormalTok{(}\DataTypeTok{parameters =}\NormalTok{ parameters_g, }\StringTok{"dgamma"}\NormalTok{)}
\end{Highlighting}
\end{Shaded}

\begin{verbatim}
## [1] "dgamma"
\end{verbatim}

\begin{verbatim}
## [1] -110.5826
\end{verbatim}

Weibull fits better.

\begin{enumerate}
\def\labelenumi{(\arabic{enumi})}
\setcounter{enumi}{2}
\tightlist
\item
  Using the Weibull model and the estimated parameters, what is the
  probability that the incubation period is shorter than 14 days?
\end{enumerate}

\begin{Shaded}
\begin{Highlighting}[]
\NormalTok{shape_w =}\StringTok{ }\KeywordTok{as.numeric}\NormalTok{(parameters_}\OperatorTok{$}\NormalTok{estimate[}\DecValTok{1}\NormalTok{])}
\NormalTok{scale_w =}\StringTok{ }\KeywordTok{as.numeric}\NormalTok{(parameters_}\OperatorTok{$}\NormalTok{estimate[}\DecValTok{2}\NormalTok{])}
\KeywordTok{pweibull}\NormalTok{(}\DecValTok{14}\NormalTok{,shape_w,scale_w)}
\end{Highlighting}
\end{Shaded}

\begin{verbatim}
## [1] 0.9994694
\end{verbatim}

What is the probability that the incubation period is longer than 5
days?

\begin{Shaded}
\begin{Highlighting}[]
\DecValTok{1}\OperatorTok{-}\KeywordTok{dweibull}\NormalTok{(}\DecValTok{5}\NormalTok{,shape_w,scale_w)}
\end{Highlighting}
\end{Shaded}

\begin{verbatim}
## [1] 0.8676643
\end{verbatim}

\begin{enumerate}
\def\labelenumi{\arabic{enumi}.}
\setcounter{enumi}{1}
\tightlist
\item
  Read the incubation period paper (posted on Moodle) and answer the
  following questions.
\end{enumerate}

\begin{enumerate}
\def\labelenumi{(\arabic{enumi})}
\tightlist
\item
  Based on results of this paper, what is the estimated median
  incubation period? Why do you think the the paper uses the median
  instead of the mean?
\end{enumerate}

\begin{enumerate}
\def\labelenumi{\alph{enumi})}
\tightlist
\item
  5.1 days (95\% CI, 4.5 to 5.8 days)
\item
  11.5 days (97.5\% CI, 8.2 to 15.6 days)
\end{enumerate}

Meadian because mean can be effected by outliers.

\begin{enumerate}
\def\labelenumi{(\arabic{enumi})}
\setcounter{enumi}{1}
\tightlist
\item
  What is the sample size of the dataset used? Discuss at least one
  reason that the dataset might be biased and not a representative
  sample.
\end{enumerate}

181 Cases sample. a) Publicly reported cases may overrepresent severe
cases, the incubation period for which may differ from that of mild
cases. b) Small sample size

\begin{enumerate}
\def\labelenumi{(\arabic{enumi})}
\setcounter{enumi}{2}
\tightlist
\item
  For the main results of the paper, what distribution is used to model
  the incubation period? Name another model mentioned as a comparison to
  the main model.
\end{enumerate}

log\_nomral distribution is assumed based of previous similar diseases.
Further more, compared with gamma, weibull and erlang distributions.

\begin{enumerate}
\def\labelenumi{\arabic{enumi}.}
\setcounter{enumi}{2}
\tightlist
\item
  All of the data and code used in the paper are published online. On
  Moodle, you can find nCoV-IDD-data-dictionary and
  nCoV-IDD-traveler-data (both are csv files that you should be able to
  open in excel). Play with these two files and answer the following
  questions.
\end{enumerate}

\begin{Shaded}
\begin{Highlighting}[]
\CommentTok{# Loading Data}
\NormalTok{covid_data =}\StringTok{ }\KeywordTok{read.csv}\NormalTok{(}\StringTok{"./nCoV_simple.csv"}\NormalTok{)}\OperatorTok{$}\NormalTok{days}
\end{Highlighting}
\end{Shaded}

\begin{enumerate}
\def\labelenumi{(\arabic{enumi})}
\tightlist
\item
  Write down what these variable names stand for: EL, ER, SL, SR, PL,
  PR.
\end{enumerate}

EL =\textgreater{} exposure left bound ER =\textgreater{} exposure right
bound SL =\textgreater{} symptom (any) onset left bound SR
=\textgreater{} (YYYY-MM-DD) symptom (any) onset right bound PL
=\textgreater{} presentation of case to hospital left bound PR
=\textgreater{} presentation of case to hospital right bound

\begin{enumerate}
\def\labelenumi{(\arabic{enumi})}
\setcounter{enumi}{1}
\tightlist
\item
  Find the case whose UID is U0021. When did he arrive in Wuhan? When
  did he leave Wuhan? If you had to make one estimate of his incubation
  period, what would it be? Briefly explain.
\end{enumerate}

1/10/2020 0:00 to 1/18/2020 23:59 About 8 days of stay.

The suspect traveled from Wuhan to Hongkon. it could have longer
incubation period as compared to mainland china. As the person may have
left without any symptoms Or may have not reported the symptoms that
could be a possible case. In gerneral, according to report people from
outside china are expected to be have little longer incubation period.

From my observarion, it clearly more then 8 days assuming that he had no
sympthoms while the person was in wuahan.

\begin{enumerate}
\def\labelenumi{(\arabic{enumi})}
\setcounter{enumi}{2}
\tightlist
\item
  Find the case whose UID is U0001. Why would it be very hard to
  determine this person's incubation period?
\end{enumerate}

This person travel Wuhan to Washington Snohomish County at 1/15/2020
11:59:00 PM. Assuming that he had no symptoms at the time of travel. We
have no data when he arrived in wuhan. Because of that we cannot bound
the possible time period of exposure in this case.

As stated in the report: ``We have used conservative bounds of possible
exposure and symptom onset where exact times were not known, but there
may be further inaccuracy in these data that we have not considered''

\begin{enumerate}
\def\labelenumi{(\arabic{enumi})}
\setcounter{enumi}{3}
\tightlist
\item
  Find another case whose incubation period would be hard to determine.
  Discuss why.
\end{enumerate}

U0003, U0004 and many more cases like this dont have left bound for
potential exposure. Alternative approches are used in these cases for
exposure esitmation. Because of the same reason as mentioned in above
question.

Other possible scenario could be, the poeple with high incubuation
period transmit the virus to the others, and we do not even know when
excatly that happend. As the first person do not show any sympthoms in
the first place. This make estimations harder.

\end{document}
